% !TeX root = ../template.tex
\hiddensection{Доказательства}

\phantomsection\label{sec_Proofs01}
\begin{proof}[Про единичный элемент]
    В \(\Matrix{n}\) выполнено \(A\Id=\Id A=A\) для любого \(A\in\Matrix{n}\).
    \begin{note}[amber]
        Используем определение \(\Id\) и стандартные свойства умножения матриц.
        \begin{note}
            Умножение на \(\Id\) не меняет строки и столбцы матрицы.
        \end{note}
    \end{note}
\end{proof}

\phantomsection\label{sec_Proofs02}
\begin{proof}[Оценка дисперсии]
    Для \(\xi\) имеем \(\Var(\xi)\ge 0\).
    \begin{note}[amber]
        \(\Var(\xi)=\E\bigl((\xi-\E\xi)^2\bigr)\ge 0\).
        \begin{note}
            Здесь используется линейность \(\E\) и неотрицательность квадрата.
        \end{note}
    \end{note}
\end{proof}


\phantomsection\label{sec_Proofs03}
\begin{proof}[Сопряжение следа]
    Для \(A\in\MatrixF{\mathbb{C}}{n}\) верно \(\Re(\tr(A))=\tr(\Re A)\).
    \begin{note}[amber]
        Это свойство покомпонентного применения \(\Re\) и линейности \(\tr\).
        \begin{note}
            Взятие вещественной части коммутирует с суммированием диагональных элементов.
        \end{note}
    \end{note}
\end{proof}