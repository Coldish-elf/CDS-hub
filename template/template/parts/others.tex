% !TeX root = ../template.tex
\hiddensection{Прочее}

\phantomsection\label{sec_Other01}
\begin{problem}[5]
    Пусть \(A\in\MatrixDim{2}{2}\) и \(A=\diag(1,-1)\).
    Найдите \( \tr(A)\) и \( \det(A)\).
    \begin{note}[green]
        Подсказка: \( \tr(\diag(\alpha,\beta))=\alpha+\beta\).
    \end{note}
\end{problem}

\phantomsection\label{sec_Other02}
\begin{solution}
    Имеем \( \tr(A)=1+(-1)=0\), а \( \det(A)= -1\).
    \begin{note}[purple]
        Также \( \spec(A)=\{1,-1\}\).
    \end{note}
\end{solution}

\phantomsection\label{sec_Other03}
\begin{question}[Тест окружений \texttt{ex}/\texttt{ans} + счётчик \texttt{\textbackslash gitem}]
    Ниже - демонстрация ручной нумерации пунктов:
    \gitem \( \Pp(\Omega)=1\).
    \gitem \( \E(\xi+\eta)=\E\xi+\E\eta\).
    \gitemreset
    \gitem После сброса счётчика снова первый пункт: \(\circled{1}\).
\end{question}

\begin{ex}
    Рассмотрим \(V=\Vector{3}\). Пусть \(U\subseteq V\) - подпространство.
    Запишите разложение \(V = U \oplus \ort(U)\).
\end{ex}

\begin{ans}
    Формально: \(V = U \oplus \ort(U)\), а проекция обозначается \( \pr_U\).
\end{ans}

\phantomsection\label{sec_Other05}
\begin{algorithm}[Мини-алгоритм с математикой]
    Вход: \(n\in\N\).
    Выход: значение \( \dec(n)\).
    \begin{note}[blue]
        Можно упомянуть лексикографический порядок \( \Lex(x,y)\) и случайную величину \(\xi\) с распределением \(\D(\xi)\).
    \end{note}
\end{algorithm}
